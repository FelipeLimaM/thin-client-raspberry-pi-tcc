% ----------------------------------------------------------
% Introdução (exemplo de capítulo sem numeração, mas presente no Sumário)
% ----------------------------------------------------------
\chapter{Introdução}
%\addcontentsline{toc}{chapter}{Introdução}
% ----------------------------------------------------------



O \textit{thin client}, ou cliente magro são computadores que utilizam o paradigma de computação centralizada, onde existe um servidor que executa serviços e programas para seus clientes. Além disso, o servidor compartilha seus recursos de \textit{hardware}, possibilitando que os computadores clientes executem \textit{softwares} sem possuir as especificações minimas desses \textit{softwares}. 

Um cliente, ou nó cliente é um computador que funciona através da rede, carregando um sistema operacional do servidor, permitindo o acesso aos  programas existentes no servidor, a utilização desse nó cliente é análoga a uma computador que funciona de maneira convencional.

Este trabalho envolve o estudo de conceitos sobre o uso do \textit{Raspberry PI} como um nó cliente. Também descreve a configuração necessária, tanto no cliente quanto no servidor para seu funcionamento. E com base nesse ambiente de teste, apresenta e discute os resultados obtidos a partir de um \textit{Raspberry PI} utilizando o sistema \textit{BerryTerminal} que o transforma em um nó cliente acessando o servidor.

O \textit{Raspberry PI} é um computador de baixo custo, que permiti um leque de opções na sua utilização. Ele possui tamanho de um cartão de credito, contem o processador, GPU e a memória RAM em um circuito integrado. É alimentado com energia de um 1 ampere e 5 Volts e possui uma massa em torno de 45 gramas. Entre as varias utilizações do \textit{Raspberry PI}, existe a utilização dele como um  um nó cliente de um rede \textit{Thin Client}.

No referencial teórico será introduzido o conhecimento necessário para um bom entendimento, possuindo uma abordagem sobre  \textit{Thin Client} e \textit{Raspberry PI}. Sobre o \textit{Thin Client} é informado sobre o que é, mostrando o principais modelo existentes, alguns aparelhos no mercado e suas aplicações bem sucedidas. 

Sobre o \textit{Raspberry PI}, contem um conteúdo abrangente, explicando o que é, seu surgimento, seus componentes em cada modelos existente e valor disponível no mercado, também informando alguns projeto que utilizam algum modelo \textit{Raspberry PI} em seu desenvolvimento.
