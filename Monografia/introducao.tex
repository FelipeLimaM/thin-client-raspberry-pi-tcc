% ----------------------------------------------------------
% Introdução (exemplo de capítulo sem numeração, mas presente no Sumário)
% ----------------------------------------------------------
\chapter{Introdução}
%\addcontentsline{toc}{chapter}{Introdução}
% ----------------------------------------------------------



O \textit{thin client} é um computador cliente totalmente dependente do servidor. O servidor compartilha seus recursos de \textit{hardware} como disco rígido (HD), Mermoria RAM (\textit{Random-Access Memory}) e CPU (\textit{Central Processing Unit}) a todos os \textit{thin clients}. Um \textit{thin client}, ou cliente magro funciona através da rede, carregando um sistema operacional (SO) do servidor, permitindo o acesso aos programas existentes no servidor, facilitando assim o \textit{backup} e a atualização dos programas utilizados e a interação com o usuário é análoga a um computador que opera de maneira convencional.

Este trabalho envolve o estudo de conceitos sobre o uso do \textit{Raspberry PI} como um \textit{thin client} e também descreve as configurações necessárias, tanto no \textit{Raspberry PI}, quanto no servidor. E com base nesse ambiente de teste, apresenta e discute os resultados obtidos a partir de um \textit{Raspberry PI} como um \textit{thin client} acessando o servidor.

O \textit{Raspberry PI} é um computador de baixo custo, que permiti um leque de opções na sua utilização. Ele possui tamanho de um cartão de crédito, contendo o processador, GPU (\textit{Graphics Processing Unit}) e a memória RAM em um circuito integrado. É alimentado com energia de um 1 ampere (A) e 5 Volts (V) e possui uma massa em torno de 45 gramas. Entre as várias utilizações do \textit{Raspberry PI}, existe a utilização dele como um  \textit{thin client}.

No referencial teórico será introduzido o conhecimento necessário para um bom entendimento, possuindo uma abordagem sobre  \textit{thin client} e \textit{Raspberry PI}, descrevendo sobre cada um, mostrando os principais produtos no mercado, e suas aplicações bem sucedidas. 

É descrito os equipamentos usados nos testes, além da descrição de como foi executados os teste apresentando todos os passos para a configuração do ambiente, afim de coletar os dados e medir o seu desempenho do \textit{Raspberry PI} na utilização visual e na utilização dos recursos do servidor, gerando resultados que são apresentados em gráficos.
